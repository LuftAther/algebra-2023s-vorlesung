\section{Halbgruppen und Monoide}
Dieses Kapitel beschäftigt sich mit elementaren Aussagen zu Halbgruppen und Monoiden. Wesentliche Resultate davon sind der \nameref*{theorem:darstellungssatz-cayley}~\ref{theorem:darstellungssatz-cayley}, der \nameref*{theorem:fundamentalsatz-der-arithmetik}~\ref{theorem:fundamentalsatz-der-arithmetik} und \cref{theorem:monoid-gruppe-einbettung}.

Zu Beginn wollen wir auf die \cref{def:halbgruppe,def:monoid,def:gruppe} hinweisen, die die im Folgenden verwendeten Begriffe \emph{Halbgruppe}, \emph{Monoid}, \emph{neutrales Element} und \emph{inverses Element} definieren.

\begin{example}
    Für eine beliebige Menge $M$ ist die Menge aller Funktionen von $M$ nach $M$ mit der Verkettung eine Halbgruppe $\mathfrak{H} = (M^M, \circ)$.
\end{example}

\begin{definition}
    Sei $\mathfrak{M} = (M, \cdot, e)$ ein Monoid und $a, a' \in M$, dann heißt
    \begin{itemize}[topsep=0pt, label={--}]
        \item $a'$ \emph{linksinvers}\index{invers!links-} zu $a$, wenn $a' \cdot a = e$ und
        \item $a'$ \emph{rechtsinvers}\index{invers!rechts-} zu $a$, wenn $a \cdot a' = e$ gilt.
    \end{itemize}
    Ist $a'$ links- und rechtsinvers zu $a$ so nennt man $a'$ \emph{invers} zu $a$ und $a$ heißt \emph{Einheit}\index{Einheit}.
\end{definition}

\begin{lemma}
    Neutrale und inverse Elemente auf Halbgruppen sind eindeutig.
\end{lemma}

\begin{proof}
    Beginnen wir mit der Eindeutigkeit von neutralen Elementen. Sei $\mathfrak{H} = (H, \cdot)$ eine Halbgruppe und seien $e, e' \in H$ neutrale Elemente. Dann gilt $e = e \cdot e' = e'.$
    
    Es bleibt noch die Eindeutigkeit von inversen Elementen zu zeigen. Sei $\mathfrak{M} = (M, \cdot, e)$ ein Monoid und seien $a, a', a'' \in M$, wobei $a'$ sowie $a''$ invers zu $a$. Wir erhalten dann $ a' = a' \cdot e = a' \cdot (a \cdot a'') = (a' \cdot a) \cdot a'' = e \cdot a'' = a''.$
\end{proof}

\begin{remark}
    Da in einem Monoid $\mathfrak{M} = (M, \cdot, e)$ immer $e \cdot e = e$ gilt, also $e$ zu sich selbst invers ist, ist $e$ immer eine Einheit. Seien $G := \{a \in M \mid a \;\text{ist Einheit von}\; \mathfrak{M}\}$ und $^{-1}: G \to G$ die Abbildung, die jedem Element sein inverses Element zuordnet, dann ist $\mathfrak{G} = (G, \cdot, e, ^{-1})$ eine Gruppe.
\end{remark}

\begin{example}
    $\mathfrak{H} = (\mathbb{R}^{2\times 2}, \cdot)$ ist eine Halbgruppe. Die Einheitsmatrix $I_2 := \left(\begin{matrix}
        1 & 0 \\ 0 & 1
    \end{matrix}\right)$ ist ein neutrales Element, womit $(\mathbb{R}^{2\times 2}, \cdot, I_2)$ ein Monoid ist. Die Menge der invertierbaren reellen $2 \times 2$ Matrizen ist die Menge aller Einheiten von $\mathfrak{H}$.
\end{example}

\begin{proposition}\label{prop:halbgruppe-monoid-einbettung}
    Sei $(H, \cdot)$ eine Halbgruppe und $e \not\in H$. Wir definieren $H' := H \cup \{e\}$ und $$\overline{\cdot}: (H')^2 \to H', (h_1, h_2) \mapsto \left\{\begin{array}{ll}
        h_1 \cdot h_2, & \;\text{wenn}\; h_1, h_2 \in H, \\
        h_1, & \;\text{wenn}\; h_1 = e, \\
        h_2, & \;\text{sonst}.
    \end{array}\right.$$

    Dann ist $(H', \overline{\cdot}, e)$ ein Monoid und es gilt $\overline{\cdot}\vert_{H^2} = \cdot$.
\end{proposition}

\begin{remark}
    Die einfach nachzurechnende \cref{prop:halbgruppe-monoid-einbettung} liefert eine einfache Möglichkeit eine Halbgruppe zu einem Monoid zu ergänzen. Sie ist der Grund, warum sich die Theorien von Halbgruppen und Monoiden sehr ähnlich sind.
\end{remark}

\begin{remark}
    Betrachten wir das freie Monoid über $X^{(1)} = \{x_1\}$. Wir erhalten damit $x_1$ als einzigen Term 0-ter Stufe, $e, x_1 \cdot x_1$ als Terme 1-ter Stufe, $e \cdot x_1, (x_1 \cdot x_1), \ldots$ als Terme 2-ter Stufe etc. Nach Faktorisieren wie in \cref{theorem:freie_faktoralgebra} erhalten wir die Repräsentanten $e, x_1, x_1^2, x_1^3, \ldots$, womit klarerweise das hier erhaltene freie Monoid kommutativ ist. Da Monoide i. A. aber nicht kommutativ sind, erhalten wir, dass freie Algebren mehr Gesetze erfüllen können, als in der gesamten Varietät gelten. 
    
    Betrachten wir allerdings das freie Monoid über $X^{(2)} = \{x_1, x_2\}$, so ist dieses nicht mehr kommutativ, also ``freier'' als das über $X^{(1)}$. 

    Ist der Generator (die Variablenmenge) $X$ mindestens abzählbar unendlich, so ist das erzeugte Monoid \emph{total frei}\index{Monoid!total frei} über $X$, also es gelten genau die Gesetze, die in der Varietät gelten.
\end{remark}

\begin{remark}
    Aus der vorherigen Bemerkung erhalten wir die folgende Beobachtung:

    Ist $\mathcal{K}$ eine Varietät, $\mathfrak{F}$ frei über $X$ in $\mathcal{K}$, dann gilt $$\forall s,t \in T(X): \mathfrak{F} \models s\approx t \Leftrightarrow \left(\forall \mathfrak{A} \in \mathcal{K}: \mathfrak{A} \models s \approx t\right).$$ 
    Ist allerdings $Y \supsetneq X$ und sind $s,t \in T(Y)$, so erhalten wir keine ähnliche Aussage über $\mathfrak{F} \models s \approx t$.
\end{remark}

\begin{theorem}[Darstellungssatz von Cayley]\label{theorem:darstellungssatz-cayley}\index{Satz!Darstellungssatz von Cayley}
    Sei $\mathfrak{M} = (M, \cdot, e)$ ein Monoid, so existiert ein injektiver Homomorphismus $\varphi: \mathfrak{M} \to (M^M, \circ, \id_M)$.
\end{theorem}
\begin{proof}
    Wähle für $a \in M$ die Funktion $f_a: M \to M, b \mapsto a \cdot b$ und sei $\varphi: M \to M^M, a \mapsto f_a$. Zeigen wir nun, dass $\varphi$ ein injektiver Homomorphismus von $\mathfrak{M}$ nach $(M^M, \circ, \id_M)$ ist. Seien $a_1, a_2 \in M$, so gilt $$\varphi(a_1 \cdot a_2) = f_{a_1 \cdot a_2} = (M \to M, b \mapsto a_1 \cdot a_2 \cdot b) = f_{a_1} \circ f_{a_2} = \varphi(a_1) \circ \varphi(a_2)$$
    und es ist $\varphi(e) = f_e = \id_M$. Damit ist $\varphi$ mit den Operationen verträglich, also ein Homomorphismus. Bleibt noch die Injektivität zu zeigen. Sei angenommen $\varphi(a_1) = \varphi(a_2)$, dann folgt daraus $a_1 = a_1 \cdot e = f_{a_1}(e) = f_{a_2}(e) = a_2 \cdot e = a_2$, womit $\varphi$ injektiv ist.
\end{proof}

\begin{theorem}[Fundamentalsatz der Arithmetik]\label{theorem:fundamentalsatz-der-arithmetik}\index{Fundamentalsatz!der Arithmetik}
    Sei $\mathfrak{S} = (S, +^{\mathfrak{S}}, 0^{\mathfrak{S}}) \le \prod_{p \in \mathbb{P}} (\mathbb{N}, +, 0)$ definiert durch
    $$ S = \{(s_p)_{p \in \mathbb{P}} \in \prod_{p \in \mathbb{P}} \mathbb{N} \mid s_p = 0 \;\text{für fast alle}\; p \in \mathbb{P}\}, $$
    dann ist $\mathfrak{S} \cong (\mathbb{N}\setminus\{0\}, \cdot, 1)$.
\end{theorem}

\begin{proof} Definieren wir $\varphi: S \to \mathbb{N}, (s_p)_{p \in \mathbb{P}} \mapsto \prod_{p \in \mathbb{P}} p^{s_p}$ und zeigen, dass dieses $\varphi$ ein Isomorphismus ist.
    \begin{itemize}[topsep=0cm, label={--}]
        \item $\varphi$ ist wohldefiniert, da für fast alle $p \in \mathbb{P}: s_p = 0$ ist und $\varphi$ damit nur auf endliche Produkte abbildet.
        
        \item Homomorphismus: Seien $(s_p)_{p \in \mathbb{P}}, (t_p)_{p \in \mathbb{P}} \in S$. Dann erhalten wir $\varphi((s_p)_{p \in \mathbb{P}} +^\mathfrak{S} (t_p)_{p \in \mathbb{P}}) = \prod_{p \in \mathbb{P}} p^{s_p + t_p} = \prod_{p \in \mathbb{P}} p^{s_p} \cdot \prod_{p \in \mathbb{P}} p^{t_p}$.
        
        \item Surjektivität: Zeigen wir mittel Induktion nach $n$ die Existenz eines Elements $\vec{s}$ aus $S$, sodass $\varphi(\vec{s}) = n$.
        
        Induktionsanfang ($n=1$): Es ist $n = \varphi(0^\mathfrak{S})$.

        Induktionsschritt ($k<n \implies n$): Ist $n \in \mathbb{P}$, so kann $\vec{s} = (\delta_{n,p})_{p \in \mathbb{P}}$ gewählt werden und damit ist $\varphi(\vec{s}) = p$. Betrachten wir nun noch den Fall $p \not\in \mathbb{P}$. Wir wissen, dass es $i, j \le n$ gibt, sodass $i \cdot j = n$. Nach der Induktionsvoraussetzung existieren $\vec s^{(i)}, \vec s^{(j)} \in S$ mit $\varphi(\vec s^{(i)}) = i$ und $\varphi(\vec s^{(j)}) = j$. Sei $\vec s := \vec s^{(i)} + \vec s^{(j)}$, dann gilt $\varphi(\vec s) = \varphi(\vec s^{(i)} + \vec s^{(j)}) = \varphi(\vec s^{(i)}) \cdot \varphi(\vec s^{(j)}) = i \cdot j = n$, weil $\varphi$ ein Homomorphismus ist.

        \notedate{29.03.2023}
        \item Injektivität: Zu zeigen ist, dass es für alle $n \in \mathbb{N} \setminus \{0\}$ höchstens eine Primfaktorenzerlegung gibt. Wir wenden Induktion nach $n$ an:
        
        Induktionsanfang ($n=1$): Klarerweise hat $1$ nur die ``triviale'' Primfaktorenzerlegung, nämlich $0 \in S$, da jedes andere Produkt echt größer als $1$ ist.

        Induktionsschritt ($k<n \implies n$): Sei indirekt angenommen $n$ hätte zwei Zerlegungen $n = p_1 \cdot ... \cdot p_e = q_1 \cdot ... \cdot q_m$, wobei $p_i, q_i \in \mathbb{P}$. Gibt es nun $i,j$ mit $p_i = q_j$, so betrachten wir
        $$ \frac{n}{p_i} = p_1 \cdot ... \cdot p_{i-1} \cdot p_{i+1} \cdot ... \cdot p_\ell = q_1 \cdot ... \cdot q_{j-1} \cdot q_{j+1} \cdot ... \cdot q_m, $$
        womit folgt, dass die Zerlegungen bereits gleich sind (bis auf Reihenfolge). Damit können wir von nun an annehmen, dass $p_i \neq q_j$ für alle $i, j$ gilt -- \obda sei $p_1 < q_1$. Wir betrachten
        $$ n' := q_1 \cdot ... \cdot q_m - p_1 \cdot q_2 \cdot ... \cdot q_m < n, $$
        so gilt insbesondere
        $$ n' = p_1 \cdot ... \cdot p_\ell - p_1 \cdot q_2 \cdot ... \cdot q_m $$
        und damit $p_1 \mid n'$. Jedoch gilt $p_1 \nmid q_1 - p_1$, da $q_1 \in \mathbb{P}$. Zerlegen wir nun
        $$ q_1 - p_1 = r_1 \cdot ... \cdot r_s $$
        in Primfaktoren, so erhalten wir
        $$ n' = (q_1 - p_1) \cdot q_2 \cdot ... \cdot q_m = r_1 \cdot ... \cdot r_s \cdot q_2 \cdot ... \cdot q_m $$
        eine Primfaktorenzerlegung von $n'$, wobei für alle $i$ $r_i \neq p_1, q_i \neq p_1$. Damit haben wir zwei verschiedene Primfaktorenzerlegungen von $n' < n$, im Widerspruch zu unserer Induktionsvoraussetzung.
    \end{itemize}
\end{proof}

\begin{remark}
    Betrachte nochmals den obigen Isomorphismus $\varphi$. Es ist $(\mathbb{N}, \leq)$ eine Totalordnung, also eine Halbordnung in der für alle $x,y$ entweder $x \leq y$ oder $y \leq x$ gilt.

    Wir definieren nun eine Halbordnung auf $S$ durch
    $$ f \leq g :\Leftrightarrow \forall p \in \mathbb{P}: f(p) \leq g(p). $$
    Mit 
    \begin{align*}
        f \vee g &:= (p \mapsto \max(f(p),g(p))), \\ f \wedge g &:= (p \mapsto \min(f(p),g(p)))
    \end{align*}
    wird $S$ also zu einem Verband $(S, \vee, \wedge)$.
\end{remark}

\begin{remark}
    Wir betrachten $(\mathbb{N} \setminus \{ 0 \}, \mid)$, wobei
    $$ n \mid k :\Leftrightarrow \exists s \in \mathbb{N}: n \cdot s = k, $$
    was eine Halbordnung bildet. Wir beobachten nun, dass für alle $f,g \in S$ gilt, dass $f \leq g \Leftrightarrow \varphi(f) \mid \varphi(g)$. Damit ist $\varphi$ ein \emph{Ordnungsisomorphismus}.
\end{remark}

\begin{corollary}
    $(\mathbb{N}, \mid)$ ist ein Verband.
\end{corollary}

\begin{proof}
    Seien $n,m \in \mathbb{N} \setminus \{0\}$ und definiere
    $$ n \vee m := \varphi(\varphi^{-1}(n) \vee \varphi^{-1}(m)) = \textrm{kgV}(n,m) $$
    $$ n \wedge m := \varphi(\varphi^{-1}(n) \wedge \varphi^{-1}(m)) = \textrm{ggT}(n,m). $$
\end{proof}

\begin{definition}\index{kürzbar!links-}\index{kürzbar!rechts-}
    Sei $H$ ein Monoid und $a \in H$. Gilt für alle $b, b' \in H$
    \begin{itemize}
        \item $ a \cdot b = a \cdot b' \implies b = b', $ so heißt $a$ \emph{linkskürzbar}.
        \item $ b \cdot a = b' \cdot a \implies b = b', $ so heißt $a$ \emph{rechtskürzbar}.
    \end{itemize}
\end{definition}

\begin{remark}
    Es stellt sich die Frage ob es möglich ist ein Monoid $(H, \cdot, e)$ in eine Gruppe einzubetten. Wir beobachten, dass in einer Gruppe für alle Elemente sowohl links-, als auch rechtskürzbar sind. Notwendig für Einbettbarkeit von einem Monoid $\mathfrak{H} = (H, \cdot, e)$ in eine Gruppe ist also jedenfalls, dass für alle $a \in H$ $a$ sowohl links- als auch rechtskürzbar ist.

    Hinreichend hingegen ist die obige Kürzbarkeit mit der zusätzlichen Forderung das $\mathfrak{H}$ kommutativ ist. Es sei angemerkt, dass, obwohl dies hinreichend ist, die Kommutativität im Allgemeinen nicht notwenig ist.
\end{remark}

\begin{example} {\ }
    \begin{enumerate}
        \item Betrachte $\Gl_2(\mathbb{R})$ und das (nicht kommutative) Untermonoid $\mathfrak{H} := \Gl_2(\mathbb{R}) \cap \mathbb{Z}^{2 \times 2}$.
        \item Betrachte die freie Gruppe über $\{x,y\}$, so erhalten wir Wörter wie $x^{n_1} y^{m_1} \cdot ... \cdot x^{n_l} y^{m_l}$ ($n_i, m_i \geq 0$).
    \end{enumerate}
\end{example}

\begin{theorem}\label{theorem:monoid-gruppe-einbettung}
    Sei $\mathfrak{H} = (H, \cdot, e)$ ein kommutatives Monoid und jedes $a \in H$ kürzbar\footnote{Aufgrund der Kommutativität reicht es sogar lediglich Links- oder Rechtskürzbarkeit zu fordern.}. Dann gilt
    \begin{enumerate}
        \item $\sim\; \subseteq (H^2)^2$
        $$ (a,b) \sim (c,d) :\Leftrightarrow a \cdot d = b \cdot c $$
        ist eine Kongruenzrelation auf $\mathfrak{H}^2$.
        \item $\mathfrak{H}^2 /_\sim$ ist eine Gruppe.
        \item Die Abbildung
        $$ \varphi :\; \mathfrak{H} \to \mathfrak{H}^2 /_\sim, \;a \mapsto [(a,e)]_\sim $$
        ist eine \emph{Einbettung}, also ein injektiver Homomorphismus.
        \item Sei $\mathfrak{G}$ eine Gruppe, so gibt es für alle $\psi : \mathfrak{H} \to \mathfrak{G}$ einen injektiven Homomorphismus $\overline{\psi} : \mathfrak{H}^2 /_\sim \to \mathfrak{G}$ mit $\overline{\psi} \circ \varphi = \psi$.
    \end{enumerate}

    \begin{figure}[H]
        \centering
        \begin{tikzpicture}
            \node[shape=ellipse,draw,minimum width=30,minimum height=30] at (0, 0) {};
            \node at (0, 0) {$\mathfrak{H}$};

            \node[shape=ellipse,draw,minimum width=70,minimum height=70] at (3, 0) {};
            \node[shape=ellipse,draw,dashed,minimum width=30,minimum height=30] at (3, 0) {};
            \node at (3, 0) {$\mathfrak{H}$};
            \node at (4.6, 1) {$\mathfrak{H}^2 /_\sim$};

            \node[shape=ellipse,draw,dashed,minimum width=30,minimum height=30] at (0.5, -2.6) {};
            \node[shape=ellipse,draw,minimum width=70,minimum height=60] at (0.5, -2.8) {};
            \node[shape=ellipse,draw,minimum width=120,minimum height=100] at (0, -3) {};
            \node at (0.5, -2.6) {$\mathfrak{H}$};
            \node at (0.5, -4.2) {$\mathfrak{H}^2 /_\sim$};
            \node at (-2.1, -2) {$\mathfrak{G}$};

            \draw [->] (0.4,0) to [bend left=15] node [midway,above] {$\varphi$} (2,0);
            \draw [->] (0,-0.4) to [bend right=15] node [midway,left] {$\psi$} (-0.3,-2.6);
            \draw [->] (3,-1) to [bend left=15] node [midway,right] {$\overline{\psi}$} (1.4,-2.8);
        \end{tikzpicture}
        \caption{Visualisierung der Einbettung von $\mathfrak{H}$ in die Gruppen $\mathfrak{G}, \mathfrak{H}^2 /_\sim$}
    \end{figure}
\end{theorem}

\begin{proof} {\ }
    \begin{enumerate}
        \item Prüfen wir zunächst, dass $\sim$ eine Äquivalenzrelation ist.
        \begin{enumerate}
            \item reflexiv: Es gilt $(a, b) \sim (a, b)$, da $a b = a b$.
            \item symmterisch: Es gilt
            $$ (a, b) \sim (c, d) \Leftrightarrow a d = b c \Leftrightarrow b c = a d \Leftrightarrow (c, d) \sim (a, b). $$
            \item transitiv: Seien $(a, b) \sim (c, d) \sim (u, v)$, es gilt also $ad = bc$ und $cv = du$. Dann folgt
            $$ (av)(cd) = addu = bcdu = (bu)(cd) $$
            und damit $av = bu$ und $(a, b) \sim (a, v)$ aus der Kürzbarkeit.
        \end{enumerate}
        Seien $(a_1, b_1) \sim (c_1, d_1), (a_2, b_2) \sim (c_2, d_2)$, also $a_1 d_1 = c_1 b_1$ und $a_2 d_2 = c_2 b_2$ und damit $a_1 a_2 d_1 d_2 = c_1 c_2 b_1 b_2$, also $(a_1 a_2, b_1 b_2) \sim (c_1 c_2, d_1 d_2)$, womit $\sim$ auch eine Kongruenzrelation ist.

        \item Wir bemerken, dass $(a, b) \sim (e, e) \Leftrightarrow ae = be \Leftrightarrow a = b$, also ist $[(e,e)]_\sim = \{ (a, a) \mid a \in H \}$ unser neutrales Element in $\mathfrak{H}^2 /_\sim$.
        
        Wegen
        $$ [(a,b)]_\sim \cdot [(b,a)]_\sim = [(ab, ab)]_\sim = [(e,e)]_\sim $$
        ist $[(b,a)]_\sim$ invers zu $[(a,b)]_\sim$, womit $\mathfrak{H}^2 /_\sim$ eine Gruppe ist.

        \item Es gilt
        $$ \varphi(e) = [(e,e)]_\sim \quad \textrm{neutral in } \mathfrak{H}^2 /_\sim, $$
        sowie für $a, b \in H$
        $$ \varphi(ab) = [(ab, e)]_\sim = [(a,e)]_\sim \cdot [(b,e)]_\sim = \varphi(a) \cdot \varphi(b), $$
        womit $\varphi$ eine Homomorphismus ist.

        Seien nun $a,b \in H$ mit $\varphi(a) = \varphi(b)$, also $[(a,e)]_\sim = [(b,e)]_\sim$, so folgt $a = ae = eb = b$, womit $\varphi$ injektiv ist.

        \item Sei \obda $\psi = \id_H$ und definiere $ \overline{\psi} :\; \mathfrak{H}^2 /_\sim \to \mathfrak{G}, \;[(a,b)]_\sim \mapsto a \cdot b^{-1} $.
        
        Seien $a,b,c,d \in H$ beliebig mit $a b^{-1} = c d^{-1}$, so folgt $ad = bc$, also $[(a,b)]_\sim = [(c,d)]_\sim$, womit $\overline{\psi}$ injektiv ist.

        Weiters ist
        \begin{align*}
            \overline{\psi}([(a,b)]_\sim \cdot [(c,d)]_\sim) &= \overline{\psi}([(ac,bd)]_\sim) = ac (bd)^{-1} = ab^{-1} \cdot cd^{-1} \\ &= \overline{\psi}([(a,b)]_\sim) \cdot \overline{\psi}([(c,d)]_\sim), 
        \end{align*}
        womit $\overline{\psi}$ ein Homomorphismus ist.
    \end{enumerate}
\end{proof}