\section{Gruppen}

\begin{definition}
    Sei $\mathfrak{G} = (G, \cdot, e, {}^{-1})$ eine Gruppe.
    \begin{itemize}
        \item Wir nennen $|G|$ die \emph{Ordnung} der Gruppe.
        \item Sei $g \in G$, so erzeugt dieses eine Untergruppe
        $$ \langle \{ g \} \rangle = \{ g^n \mid n \in \mathbb{Z} \}. $$
        Wir nennen $|\langle\{g\}\rangle|$ die \emph{Ordnung} von $g$ und schreiben auch $\ord(g)$. Ist $\ord(g)$ endlich, so heißt $g$ \emph{Torsionselement}.
        \item $\mathfrak{G}$ heißt \emph{zyklisch}, falls es ein $g \in G$ mit $G = \langle\{g\}\rangle$ gibt.
    \end{itemize}
\end{definition}

\begin{example}
    
\end{example}

\begin{example}
    Die Gruppen $(\mathbb{Z}, +, 0, -) = \langle\{1\}\rangle, (\mathbb{Z}_m, +, 0, -) = \langle\{1\}\rangle$ sind zyklisch.

    Die Gruppe $(\Gl_2(\mathbb{Q}), \cdot, E_2, {}^{-1})$ ist nicht zyklisch.
\end{example}